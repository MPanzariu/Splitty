\section{DEALING WITH CONFLICTS}
In the context of a team, views and opinions between team members may differ. Sometimes, those views can differ so much that conflicts may arise. One of the most important aspects of teamwork is knowing how to deal with said conflicts.\par 
\smallskip
We found two types of conflict that can occur: conflicts involving parts of the project and personal conflicts. The first type of conflict can appear when two or more people have a problem with a certain aspect of the project, such as: the look of the overall application, the ways to implement certain functionalities etc. In that case, we must decide on the correct way to proceed. This kind of conflict is solved by casting a majority vote with arguments (it is part of the category decisions of change). First, the two people involved in the project present their points of view and then, the other team-members cast their votes. \par
\smallskip
In the case of the second type of conflict, things can get more complicated. Personal conflicts can appear because of different views of people on aspects outside the project environment, such as: political views, racial issues etc. Although these problems should not be a concern for the team, it is important to solve them to maintain a healthy working environment. Since these matters are sensitive, we cannot just vote for the correct way to handle the issue. Personal matters should be solved by the two parties on their own. \par
\smallskip
In case conflicts escalate, for example someone does not agree with the vote of the other teammates, we need to find a way to solve the problem. Since we could not convince the other person, the best course of action is to address the problem to the Teacher assistant and hope that he can offer insight on the matter, such that everyone in the team can understand the mistake and continue collaborating.\par 
