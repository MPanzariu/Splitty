\section{Behavior}
\subsection{General Behavior}
To facilitate our teamwork, we want to foster a friendly and productive environment where anyone is allowed to make mistakes. To achieve this we want to be forgiving to each other. When someone is annoyed by someone's behavior, then it is important to apply the AID model. This allows us to respectfully communicate with each other when there are tensions.

% Everyone has their own perspectives, private situations which could be troubling that might lead to unproductive behavior for the project. Nobody has the obligation to share such details with each other in its entirety, which is why it is important to remain polite to remain respectful to one another. 

% In the case that perceived mistakes happen, it is important to remain forgiving to each other. Everyone makes mistakes, or just has their own disadvantages to deal with. If we are not forgiving to each other, we only damage the social and productive atmosphere that we aim to keep up.
\subsection{Disagreements}
Here we lay out a plan for handling disagreements that will inevitably happen during the span of the project: These disagreements can be about trivial design/coding decisions, ambiguity related questions, or the effect of personal problems. In essence, any kind of disagreement.
\begin{enumerate}
    \item Everyone that is involved in the conflict/disagreement ought to seriously listen to each other. This means giving other people the time they need, to bring forth their opinion. To take the time yourself to think deeply about the brought up opinions. It is key to be aware of the fact that everyone has their own valid perspective on topics. When everyone is heard, it is important to try to come together to find a solution.
    \item In the case that involved participants aren’t able to find common ground, then the next link in the chain will become the entire group. Here, essentially the same recipe as has been stated before will be repeated; listen to everyone, understand the different perspectives, and find a solution. It is vital that everyone accepts at this stage that compromises might have to be made, because the next stages of escalation are only meant for the problems of utmost importance.
    \item Sometimes, problems cannot be resolved with the aforementioned procedures. In such a scenario it is best to have a proper discussion with our project TA. During this discussion, the following points should be introduced to the TA: all the aspects of the problem, every side maintained by any party involved, and the measures that the group has gone through to get to this point. Advice of the TA should be taken very seriously, because the proper handling of these problems should have been our responsibility.
    \item When all other measures are exhausted, contact with the head TA will become necessary. Here the same procedure for contact with our project TA will be applied. Be aware that this is only meant as a last resort.
\end{enumerate}
\subsection{Meeting Punctuality}
The actual start of meetings will depend on the punctuality of its members.
For the weekly TA meeting:
\begin{itemize}
    \item If some member is late, and has been late for at least 2 times before, then the start of the meeting won’t take the presence of this person into account.
    \item Else if someone communicates before the slotted time that he will be 5 to 10 minutes late, then wait for a maximum of 5 minutes to start the meeting.
    \item Else start the meeting at the beginning of our time slot.
During additional meetings we will wait 15 minutes, if someone communicates before the agreed upon start time of the meeting that he will be late for no more than 15 minutes.
\end{itemize}